\documentclass[11pt,a4paper]{article}

\usepackage[applemac]{inputenc} %Careful with that Captain Ol.
\usepackage{latexsym}
\usepackage{graphics}
\usepackage[francais]{babel}
\usepackage{amsmath,amssymb}
\usepackage{pstricks,pst-plot}
\usepackage{calc}
\usepackage{multicol}
\usepackage{fancyhdr}
\usepackage{lastpage}
\usepackage[T1]{fontenc}
\usepackage{stmaryrd}
\pagestyle{plain}



\begin{document}

\section{Learning in discrete graphical models}

Let $(x_1, z_1), \ldots, (x_N, z_N)$ be an i.i.d. sample of observations. We will denote $n_m$ the number of $z_i$s which are equal to $m$ ($m \in \llbracket 1, M \rrbracket$). We have $\sum\limits_{m=1}^{M} n_m = N$.
\\We compute the likelihood for Z: $\mathcal{L} = \mathop{\Pi}\limits_{i=1}^N p(Z = z_i) = \mathop{\Pi}\limits_{m=1}^M \pi_m^{n_m}$.
\\Thus the log likelihood is equal to $\ell = \mathcal{L} = \sum\limits_{m=1}^M n_m \mathrm{log} \pi_m$.
\\We want to maximize this quantity under the constraint that $\sum\limits_{m=1}^{M} \pi_m = 1$. Using a Lagrange multiplier method, we have that there exists $\lambda$ such that $$\forall m \in \llbracket 1, M \rrbracket, \frac{n_m}{\pi_m} = \lambda \times 1$$
that is $$\forall m \in \llbracket 1, M \rrbracket, n_m = \lambda \pi_m$$
if we sum these $m$ equalities, since $\sum\limits_{m=1}^{M} n_m = N$, we get $\lambda = N$.
So the likelihood is maximal iff $$\forall m \in \llbracket 1, M \rrbracket,  \pi_m = n_m/N$$
which is a very natural result : the estimator for $p(Y=m)$ is just the proportion of observations $z_i$ equal to $m$.
\\[5mm]The calculation is a bit more complicated for $\theta$, but the idea is the same.
\\In a similar fashion, for $k \in \llbracket 1, K \rrbracket$ and $\llbracket 1, M \rrbracket$ respectively, we denote $p_{km}$ the number of $x_i$s equal to $k$, and whose label is equal to $m$. 
\\We must have $\forall m, \,  \sum\limits_{k=1}^{K}  p_{km} = n_m$
\\We compute the likelihood for $\theta$ : 
$$\begin{aligned} \mathcal{L} &= \mathop{\Pi}\limits_{i=1}^N p(X = x_i) \\
 &= \mathop{\Pi}\limits_{i=1}^N \sum\limits_{k=1}^K p(X=x_i | Z= k) p(Z=k)\\
 & = \mathop{\Pi}\limits_{i=1}^N  \sum\limits_{k=1}^K \theta_{x_i k} \frac{n_k}{N} \\
 & = \frac{1}{N^N} \mathop{\Pi}\limits_{i=1}^N  \sum\limits_{k=1}^K \theta_{x_i k} n_k \\
  & = \frac{1}{N^N} \mathop{\Pi}\limits_{m=1}^M ( \sum\limits_{k=1}^K \theta_{m k} n_k )^{p_m}
 \end{aligned}$$
We must maximize this with the constraint that the sum of all $\theta_{m k}$s must be $K$.
\\Again we use a Lagrange multiplier.


\end{document}